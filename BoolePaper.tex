\documentclass[12]{article} %the sets the document type.

\usepackage{amsmath,amssymb,amsthm} %this loads commonly used math packages
\usepackage{enumitem}
\usepackage{float}
\newtheorem{theorem}{Theorem} %this declares the theorem environment
\setlist[enumerate]{leftmargin=*,widest=0}
\usepackage{graphicx}
\graphicspath{ {ApolloniusResources/} }

\title{George Boole}
\author{Nick Zayatz and Michele Burns}

\renewcommand{\baselinestretch}{1.5} 
\begin{document} %document starts here
\maketitle % makes the title

Over the course of his life, George Boole made some major contributions to mathematics. In the field of analysis, Boole published his treatise \textit{Applications to the Theory of Definite Integrals On the Comparison of Transcendents, with Certain Applications to the Theory of Definite Integrals} in 1857.

Despite his contributions to other mathematical fields, Boole most important addition to the field of mathematics was his work involving symbolic logic, which many now refer to as "Boolean Algebra." Boole first introduced this topic of study in his treatise "The Mathematical Analysis of Logic" in 1847. The idea was then further expanded and refined in 1854, when Boole published his most famous work "An Investigation of The Laws of Thought."

\newpage



\begin{thebibliography}{9}
\bibitem{AnalysisBook} 
Boole, George. 
\textit{Applications to the Theory of Definite Integrals On the Comparison of Transcendents, with Certain Applications to the Theory of Definite Integrals}. 
London: Philosophical Transactions, 1857. 745-803. Print.
 
\bibitem{BooleanBook2} 
Boole, George. 
\textit{An Investigation of the Laws of Thought: On Which Are Founded Mathematical Theories of Logic and Probabilities}.
New York: Dover, 1854. Print.
 
\bibitem{MacTutor} 
"George Boole."
\textit{Boole Biography}.
JOC/EFR, June 2004. Web. 16 Feb. 2016. 
\end{thebibliography}

\end{document}